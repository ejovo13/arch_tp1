% This document was prepared to replicate the style of the Algorithmique
% Generale TP structure.

% \vspace{-10pt}
% This is the PERFECT SCALING, now we need to adjust the scale a little bit.
\begin{tikzpicture}[remember picture,overlay,yshift=-1.2cm, xshift=1.75cm] % LMAOOOOOO This is the EXACT POSITION!!!!
    \node at (0,0) {\includegraphics[width=4.0cm,height=1.6cm]{media/1280px-Logo_Polytech_Sorbonne.png}};
\end{tikzpicture}

% \hspace{-5cm}
% \hspace{-1cm}
% \hspace{-1cm}

% \vspace{-1cm}
\vspace{0.3cm}

% {\raggedleft \color{mygold} Programmation en Python\\
% MAIN-3, année 2022\\
% Séance TP N\degree 1\\
% Février 2022\\
% VOYLES Evan\\}

{\raggedleft Printemps 2022\\
RAKOTOVAO Jonathan\\
VOYLES Evan\\}

% This vspace accomodates my name
\vspace{-0.42cm}
\vspace{1.23cm}

{\centering \Large \textbf{Circuits logiques}\par}
\vspace{0.4cm}
{\centering \large \textbf{TP 1}\par}

\noindent\rule{\textwidth}{1pt}


% {\Large \noindent \color{mygold} Objectif}

% {\color{mygold}\noindent\rule{\textwidth}{1pt}}
% \vspace{0cm}
% \begin{itemize}
%     \item[{\color{mygold}\ding{43}}] Expressions régulières.
% \end{itemize}

% \vspace{1.2cm}
% {\color{dullgreen}\noindent \Large \bf Problème}

% \vspace{-0.28cm}
% {\vspace{0cm}\color{dullgreen}\noindent\rule{\textwidth}{1pt}}

% \begin{textblock*}{10cm}(13.58cm,7.2cm) % {block width} (coords)
%     \Large \aspurp{[Expression régulières]}
% \end{textblock*}
