\usepackage[T1]{fontenc}
\usepackage[french]{babel}
\usepackage[utf8]{inputenc} % set input encoding (not needed with XeLaTeX)
% \usepackage{kpfonts}

%%% Examples of Article customizations
% These packages are optional, depending whether you want the features they provide.
% See the LaTeX Companion or other references for full information.

%%% PAGE DIMENSIONS
\usepackage{geometry} % to change the page dimensions
\geometry{a4paper} % or letterpaper (US) or a5paper or....
% THE MARGINS FOR DSA is 1.5cm
% \geometry{margin=1.5cm} % for example, change the margins to 2 inches all round
\geometry{margin=1cm} % for example, change the margins to 2 inches all round
\geometry{right = 1cm}
% \geometry{landscape} % set up the page for landscape
%   read geometry.pdf for detailed page layout information


% \usepackage[parfill]{parskip} % Activate to begin paragraphs with an empty line rather than an indent

%%% PACKAGES
\usepackage{booktabs} % for much better looking tables
\usepackage{array} % for better arrays (eg matrices) in maths
\usepackage{paralist} % very flexible & customisable lists (eg. enumerate/itemize, etc.)
\usepackage{verbatim} % adds environment for commenting out blocks of text & for better verbatim
% \usepackage{subfig} % make it possible to include more than one captioned figure/table in a single float
% These packages are all incorporated in the memoir class to one degree or another...

%%% HEADERS & FOOTERS
\usepackage{fancyhdr} % This should be set AFTER setting up the page geometry
% \pagestyle{fancy} % options: empty , plain , fancy

\usepackage{graphicx} % support the \includegraphics command and options
\usepackage{subcaption}
\usepackage{caption}
% \usepackage{tikz}

% \usepackage{uftsym}
\usepackage{dingbat} % For the pointy hands
\usepackage{pifont}
% \usepackage{xcolor} % For pretty colors
\usepackage[table]{xcolor}
\usepackage{tikz} % for nice pictures
\usepackage{blindtext}
\usepackage{wrapfig}
\usepackage{gensymb}
% \usepackage{table}

% COLORs
\definecolor{mygold}{RGB}{182, 153, 45}
\definecolor{mygreen}{RGB}{62, 171, 0}
\definecolor{dullgreen}{RGB}{165, 181, 45}
\definecolor{mypurp}{RGB}{84, 45, 181}
\definecolor{lpurp}{RGB}{146, 45, 203}
\definecolor{lgreen}{RGB}{143, 209, 68}


% \renewcommand{\headrule}{\color{gray}}
\renewcommand{\headrule}{\hbox to\headwidth{%
  \color{gray}\leaders\hrule height \headrulewidth\hfill}}

\renewcommand{\footrulewidth}{1pt}
% \renewcommand{\footrule}{\hbox to\headwirth{
%     \color{gray}\leaders\hrule height \footrulewidth\hfill}}

\renewcommand{\footrule}{{\color{gray}\vskip-\footruleskip\vskip-\footrulewidth \hrule width\headwidth height\footrulewidth\vskip\footruleskip}}

\fancyhf{}
\rhead{\textcolor{gray}{Séance TP 1}}
\chead{\color{gray} Travaux Pratiques}
% \lhead{Optimisation du GCC}
\lfoot{\color{gray}\textcopyright 2022 Evan Voyles}
\rfoot{\color{gray} Page \thepage\ sur 3}
\cfoot{\color{gray}Spécialité MAIN-3}
\footskip = 0pt
% \voffset = 10pt
% \headsep = 0pt
% \cfoot{\thepage\ of \pageref{LastPage}}

% \renewcommand{\headrulewidth}{0pt} % customise the layout...
% \lhead{}\chead{}\rhead{}
% \lfoot{}\cfoot{\thepage}\rfoot{}


\usepackage[absolute,overlay]{textpos} % Add text in any arbitrary position

%%% SECTION TITLE APPEARANCE
\usepackage{sectsty}
\allsectionsfont{\sffamily\mdseries\upshape} % (See the fntguide.pdf for font help)
% (This matches ConTeXt defaults)

%%% ToC (table of contents) APPEARANCE
\usepackage[nottoc,notlof,notlot]{tocbibind} % Put the bibliography in the ToC
\usepackage[titles,subfigure]{tocloft} % Alter the style of the Table of Contents
\renewcommand{\cftsecfont}{\rmfamily\mdseries\upshape}
\renewcommand{\cftsecpagefont}{\rmfamily\mdseries\upshape} % No bold!

%%% END Article customizations

%%% The "real" document content comes below...

\newcommand{\asgold}[1]{\textcolor{mygold}{{\bf#1}}}
\newcommand{\asgrey}[1]{\textcolor{gray}{{\bf#1}}}
\newcommand{\asred}[1]{\textcolor{red}{{\bf#1}}}
\newcommand{\asor}[1]{\textcolor{orange}{{\bf#1}}}
\newcommand{\ascy}[1]{\textcolor{cyan}{{\bf#1}}}
\newcommand{\asgr}[1]{\textcolor{mygreen}{{\bf#1}}}
\newcommand{\aspurp}[1]{\textcolor{mypurp}{{\bf#1}}}
\newcommand{\aslprp}[1]{\textcolor{lpurp}{{\bf#1}}}
\newcommand{\aslgrn}[1]{\textcolor{lgreen}{{\bf#1}}}
